\documentclass[plainsections,  36pt]{sciposter}
\usepackage[english]{babel}
\usepackage{xcolor,eso-pic,graphicx,wallpaper}
\usepackage{fix-cm} % Huge
\usepackage{amsfonts,wrapfig} %simbolos matematicos
\usepackage{amssymb,amsmath,amsthm } %simbolos matematicos
\usepackage{textcomp}
\usepackage{psfrag, color,ifpdf,multicol,xspace}
\usepackage[utf8]{inputenc}

%\renewcommand{\sfdefault}{verdana}
%\usepackage{verbatim}
\usepackage{graphicx}%
\usepackage{color}
%\usepackage[small,bf]{caption}

%\usepackage{gtamacverdana}
%\usepackage{verdana}

\providecommand{\U}[1]{\protect\rule{.1in}{.1in}}
%EndMSIPreambleData
\setcounter{tocdepth}{4}

\renewenvironment{proof}[1][Demonstração:]{\noindent\textbf{#1} }{\hfill   \rule{0.5em}{0.5em} \vspace{1cm}}
\newenvironment{example}[1][Exemplo]{\addtocounter{theorem}{1} \noindent\textbf{#1 \arabic{theorem}.} }{\hfill \rule{0.5em}{0.5em}  \vspace{1cm}}
\newenvironment{examples}[1][Exemplos]{\addtocounter{theorem}{1} \noindent\textbf{#1 \arabic{theorem}.} }{\hfill  \rule{0.5em}{0.5em} \vspace{1cm}}
%\newtheorem*{example}{Exemplo}
\newcounter{theorem}
\newtheorem*{acknowledgement}{Acknowledgement}
\newtheorem*{axiom}{Axiom}
\newtheorem*{case}{Case}
\newtheorem*{claim}{Claim}
\newtheorem*{conclusion}{Conclusion}
\newtheorem*{condition}{Condition}
\newtheorem*{conjecture}{Conjecture}
\newtheorem*{corollary}{Corollary}
\newtheorem*{criterion}{Criterion}
\newtheorem*{defi}{Definition}
%\newtheorem*{example}{Exemplo}
%\newtheorem*{examples}{Exemplos}
\newtheorem*{exercise}{Exercise}
\newtheorem*{lemma}{Lema}
\newtheorem*{lema}{Lema}
\newtheorem*{notation}{Notation}
\newtheorem*{problem}{Problem}
\newtheorem*{prop}{Proposition}
\newtheorem*{remark}{Remark}
\newtheorem*{solution}{Solution}
\newtheorem*{summary}{Summary}
\newtheorem*{teo}{Theorem}
\newtheorem*{obser}{Observação}

%opening
\usepackage{babel}


\definecolor{amarelo}{HTML}{FFCC00}
\definecolor{verde}{HTML}{006600}
\renewcommand{\papertype}{custom}
\setlength{\paperwidth}{90cm}
\setlength{\paperheight}{100cm}
\renewcommand{\setpspagesize}{
  \ifthenelse{\equal{\orientation}{portrait}}{
    \special{papersize=90cm,100cm}
    }{\special{papersize=90cm,100cm}
  }
}
\setlength{\topmargin}{0in}
\setlength{\headheight}{0in}
\setlength{\headsep}{0in}
\setlength{\textheight}{84cm}
\setlength{\textwidth}{75.5cm}%{80cm}
\setlength{\oddsidemargin}{2.4cm}
\setlength{\evensidemargin}{0cm}
\setlength{\parindent}{0in}%{0.25in}
\setlength{\parskip}{0.25in}
\setlength{\pdfpagewidth}{90cm}
\setlength{\pdfpageheight}{100cm}
\newcommand\BackgroundPic{
  \put(-82,65){
    \parbox[b][\paperheight]{\paperwidth}{%
      \vfill
      \centering
      \includegraphics[width=\paperwidth,height=\paperheight,
      keepaspectratio]{poster-sic.jpg}%
      \vfill
    }
  }
}
\setlength{\columnseprule}{0pt}

\renewcommand{\titlesize}{\fontsize{84}{35}\selectfont }
\newcommand{\largo}{\fontsize{36}{40}\selectfont }
\makeatletter
% La commande \section est définie comme suit
\renewcommand\section{\@startsection {section}{1}{\z@}%
                                   {-2ex \@plus -1ex \@minus -.1ex}%
                                   {0.8ex \@plus.1ex}%
                                   {\normalfont\largo\bfseries}}
\makeatother
\vspace{-1cm}

\def\thesection{}

%%%%%%%%%%%%%%%%%%%%%%%%%%%%%%%%%%%%%%%%%%%%%%%%%%%%%%%%%%%%%%%%%%%
%%%%%%%%%%%%%%%%%%%%%%%%%%%%%%%%%%%%%%%%%%%%%%%%%%%%%%%%%%%%%%%%%%%
%%%%%%%%%%%%%%%%%%%%%%%%%%%%%%%   Título   %%%%%%%%%%%%%%%%%%%%%%%%%%%%%%%%%%%%%5
\title{
  \vspace{-3cm} 
  \hspace{-27cm}
  \textcolor{amarelo}{
    \parbox{0.9\textwidth}{
      \textsc{Um sistema para auxiliar na\\ 
        aprendizagem da disciplina\\ 
        Linguagens Formais e Autômatos
      }
    }
  }
}
%%%%%%%%%%%%%%%%%%%%%%%%%%%%%%%%%%%%%%%%%%%%%%%%%%%%%%%%%%%%%%%%%%%
%%%%%%%%%%%%%%%%%%%%%%%%%%%%%%%%%%%%%%%%%%%%%%%%%%%%%%%%%%%%%%%%%%%
%%%%%%%%%%%%%%%%%%%%%%%%%%%%%%%%%%%%%%%%%%%%%%%%%%%%%%%%%%%%%


\begin{document}


\AddToShipoutPicture{\BackgroundPic}
\makeatletter
\AddToShipoutPicture{%
  \setlength{\@tempdimb}{.5\paperwidth}%
  \setlength{\@tempdimc}{.5\paperheight}%
  \setlength{\unitlength}{1pt}%
  \put(\strip@pt\@tempdimb,\strip@pt\@tempdimc){%
  }%
}
\makeatother
\maketitle
%\mbox{}\vspace{6cm}

\begin{center}
\textbf{ \large Rafael Cardoso da Silva }\\
Centro de Matemática, Computação e Cognição, Universidade Federal do ABC\\
Av. dos Estados, 5001, Santo André, SP\\
rafael.cardoso@aluno.ufabc.edu.br
\end{center}

\vspace{1.2cm}
\begin{multicols}{2}
  
\paragraph{Resumo:}
  Resolver exercícios é fundamental para um aluno fixar os conceitos apresentados em aula. Por outro lado, ter seus exercícios corrigidos também é muito importante, para que ele possa avaliar o seu aprendizado. Na UFABC, a disciplina de Linguagens Formais e Autômatos contempla vários exercícios que admitem infinitas respostas, o que torna a correção deles praticamente impossível, principalmente quando as turmas são grandes. O objetivo deste projeto foi a criação e implementação de um sistema para aplicação e correção automática de exercícios envolvendo autômatos finitos determinísticos. Através do estudo de métodos e algoritmos presentes na literatura, foi possível implementar o teste de equivalência entre o autômato-resposta do aluno e o autômato-gabarito previamente armazenado no banco de dados. Ao final do projeto, o sistema foi usado em caráter experimental numa turma da UFABC da disciplina de Linguagens Formais e Autômatos, afim de testar a sua qualidade. E ao final da disciplina, a nota que os alunos obtiverem ao revolver os exercícios do sistema ajudarão a compor o conceito final de cada um na disciplina.

  \noindent \textbf{Palavras-chave:} autômato finito, equivalência de autômatos, minimização de autômato, programação para web.

\section{Introdução}

  


\section{Metodologia}
- Moore

- HKE

- DFAjudge



\section{Conclusão}



\section{Referências}
// apenas citada neste poster

As referências devem ser feitas respeitando-se as normas definidas pela ABNT. Somente deverão ser apresentadas as referências citadas no referido pôster.

Nas referências no texto citar o nome do autor e o ano da publicação (SANTOS, 2003).

SANTOS, Bruno A. Aspectos conceituais e arquiteturais para a criação de linhagens de agentes de software cognitivos e situados. 2003. 130f. Dissertação (Mestrado em Tecnologia – Manufatura Integrada por Computador) – Centro Federal de Educação Tecnológica de Minas Gerais, Belo Horizonte, 2003.


HOPCROFT, J. An $n~log~n$ algorithm for minimizing states in a finite automaton. In:
\textit{Theory of machines and computations (Proc. Internat. Sympos., Technion, Haifa, 1971).} [S.l.]: Academic Press, New York, 1971. p. 189–196.

HOPCROFT, J.; KARP, R. A Linear Algorithm for Testing Equivalence of Finite Automata. [S.l.], 1971.

HOPCROFT, J. E.; KARP, R. M. 
A Linear Algorithm for Testing Equivalence of Finite Automata.
A linear time algorithm for testing equivalence of finite automata. 
\textit{Technical report of Cornell University}, p. 71–114, 1971.

%%%%%Deve conter agradecimento agência de fomento
\vspace{1cm}
\noindent\textbf{Este trabalho foi financiado pelo Programa de Iniciação Científica da UFABC. }

\end{multicols}
\end{document}